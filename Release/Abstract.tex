\documentclass[12pt]{article}
\usepackage{graphicx}
%\usepackage{subcaption}
\usepackage{caption}
\usepackage{mathtools}
\usepackage{tikz,pgfplots}
\usepackage{subfig}
\usepackage{epsfig}
\usepackage{amsmath}
\usepackage{amssymb}
\usepackage[shortlabels]{enumitem}
\usetikzlibrary{angles,patterns,calc}
\usepackage{bbm}
\usepackage{float}
\newcommand\der[2]{\frac{\partial{#1}}{\partial{#2}}}
\DeclareMathOperator*{\argmax}{arg\,max}
\DeclareMathOperator*{\argmin}{arg\,min}

% stuff to put matlab code in 
\usepackage{listings}
\usepackage{color} %red, green, blue, yellow, cyan, magenta, black, white
\definecolor{mygreen}{RGB}{28,172,0} % color values Red, Green, Blue
\definecolor{mylilas}{RGB}{170,55,241}

% Shortcut greek
\def\a{\alpha}
\def\b{\beta}
\def\g{\gamma}
\def\D{\Delta}
\def\d{\delta}
\def\z{\zeta}
\def\k{\kappa}
\def\l{\lambda}
\def\n{\nu}
\def\r{\rho}
\def\s{\sigma}
\def\t{\tau}
\def\x{\xi}
\def\w{\omega}
\def\W{\Omega}

\usepackage[utf8]{inputenc}
\usepackage[english]{babel}
\usepackage{fancyhdr}
\fancypagestyle{firststyle}
{
\fancyhf{}
    \renewcommand{\headrulewidth}{0pt}
   \fancyfoot[C]{\footnotesize Page \thepage\ of \pageref{LastPage}}
}

\newcommand{\numpy}{{\tt numpy}}    % tt font for numpy

\topmargin -.5in
\textheight 9in
\oddsidemargin -.25in
\evensidemargin -.25in
\textwidth 7in

\newcommand{\question}[1]{ \begin{center} \noindent\colorbox{gray!10}{
\parbox{0.8\textwidth}{\vspace{0.125in} #1 \vspace{0.125in} } } \end{center} }

\begin{document}

\thispagestyle{firststyle}

\author{Isaac Liu, Nicol\'as Martorell \& Paul Opheim}
\title{Attenuation Bias, Measurment Error \& Principal Component Analysis} 
\maketitle

% code

\lstset{language=Matlab,%
    %basicstyle=\color{red},
    breaklines=true,%
    morekeywords={matlab2tikz},
    keywordstyle=\color{blue},%
    morekeywords=[2]{1}, keywordstyle=[2]{\color{black}},
    identifierstyle=\color{black},%
    stringstyle=\color{mylilas},
    commentstyle=\color{mygreen},%
    showstringspaces=false,%without this there will be a symbol in the places where there is a space
    numbers=left,%
    numberstyle={\tiny \color{black}},% size of the numbers
    numbersep=9pt, % this defines how far the numbers are from the text
    emph=[1]{for,end,break},emphstyle=[1]\color{red}, %some words to emphasise
    %emph=[2]{word1,word2}, emphstyle=[2]{style},    
}

\begin{abstract}
    Many variables of interest in Economics are not directly available as empirical data. Instead, economists often use other variables that are imperfect measurements of the true focus of their analysis. These available variables are known as \textit{proxies} or ``variables measured with error'', and, if they suffer from classical measurement error, their use causes \textit{attenuation bias} when they are used as independent variables in econometric estimation. Traditionally, instrumental variables are used as a shock of exogeneity to get rid of this bias, but finding truly exogenous variables that satisfy the exclusion restriction is difficult, and so this method can often not be feasibly applied.\\
    \\
    As an alternative to dealing with attenuation bias, we propose the use of Principal Component Analysis (PCA) over several variables measured with error. When there are multiple observed variables driven by a single ``true'' one, we propose to use PCA over these variables to extract the ``true'' variable. We then use this extracted value and use it in a standard OLS regression, thus providing a solution to attenuation bias that does not require the strong assumptions of instrumental variable analysis.\\
    \\
    To show the properties and behaviour of our estimator on large samples under standard assumptions, we present a theoretical framework and a Monte-Carlo analysis. Additionally, we explore a basic empirical application to our method, by estimating the effect of economic development on life expectancy at birth. Since there is no consensus on how to measure economic development, we take a sample of different variables that may measure economic development with error (GDP per capita, GNI per capita, Household Income Per Capita, among others) over which we apply PCA to apply our identification strategy.
\end{abstract}

\end{document}