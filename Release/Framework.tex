\documentclass[12pt]{article}
\usepackage{graphicx}
%\usepackage{subcaption}
\usepackage{caption}
\usepackage{mathtools}
\usepackage{tikz,pgfplots}
\usepackage{subfig}
\usepackage{epsfig}
\usepackage{amsmath}
\usepackage{amssymb}
\usepackage[shortlabels]{enumitem}
\usetikzlibrary{angles,patterns,calc}
\usepackage{bbm}
\usepackage{float}
\newcommand\der[2]{\frac{\partial{#1}}{\partial{#2}}}
\DeclareMathOperator*{\argmax}{arg\,max}
\DeclareMathOperator*{\argmin}{arg\,min}

% stuff to put matlab code in 
\usepackage{listings}
\usepackage{color} %red, green, blue, yellow, cyan, magenta, black, white
\definecolor{mygreen}{RGB}{28,172,0} % color values Red, Green, Blue
\definecolor{mylilas}{RGB}{170,55,241}

% Shortcut greek
\def\a{\alpha}
\def\b{\beta}
\def\g{\gamma}
\def\D{\Delta}
\def\d{\delta}
\def\z{\zeta}
\def\k{\kappa}
\def\l{\lambda}
\def\n{\nu}
\def\r{\rho}
\def\s{\sigma}
\def\t{\tau}
\def\x{\xi}
\def\w{\omega}
\def\W{\Omega}

\usepackage[utf8]{inputenc}
\usepackage[english]{babel}
\usepackage{fancyhdr}
\fancypagestyle{firststyle}
{
\fancyhf{}
    \renewcommand{\headrulewidth}{0pt}
   \fancyfoot[C]{\footnotesize Page \thepage\ of \pageref{LastPage}}
}

\newcommand{\numpy}{{\tt numpy}}    % tt font for numpy

\topmargin -.5in
\textheight 9in
\oddsidemargin -.25in
\evensidemargin -.25in
\textwidth 7in

\newcommand{\question}[1]{ \begin{center} \noindent\colorbox{gray!10}{
\parbox{0.8\textwidth}{\vspace{0.125in} #1 \vspace{0.125in} } } \end{center} }

\begin{document}

\thispagestyle{firststyle}

\author{Isaac Liu, Nicol\'as Martorell \& Paul Opheim}
\title{Attenuation Bias, Measurment Error \& Principal Component Analysis} 
\maketitle

% code

\lstset{language=Matlab,%
    %basicstyle=\color{red},
    breaklines=true,%
    morekeywords={matlab2tikz},
    keywordstyle=\color{blue},%
    morekeywords=[2]{1}, keywordstyle=[2]{\color{black}},
    identifierstyle=\color{black},%
    stringstyle=\color{mylilas},
    commentstyle=\color{mygreen},%
    showstringspaces=false,%without this there will be a symbol in the places where there is a space
    numbers=left,%
    numberstyle={\tiny \color{black}},% size of the numbers
    numbersep=9pt, % this defines how far the numbers are from the text
    emph=[1]{for,end,break},emphstyle=[1]\color{red}, %some words to emphasise
    %emph=[2]{word1,word2}, emphstyle=[2]{style},    
}

\section{Theoretical framework}

Consider a model where the outcome is denoted by $y_i$. This outcome depends on a variable of interest denoted by $t_i$ and a vector of covariates denoted by $X_i=(x_{i,1},x_{i,2},\dots x_{i,p})'$. Additionally, consider a vector of variables $X^*_i=(x^*_{i,1},x^*_{i,2},\dots x^*_{i,p})'$ that correspond to the covariates $X_i$ but observed with measurement error, where $x^*_{i,k}=x_{i,k}+\eta_{i,k}$ with $\eta_{i,k} \sim {iid}(0,\sigma^2_{\eta_k})$, $\operatorname{E}(x_{i,k}'\eta_{i,k})=0, \forall i$, $\operatorname{E}(x_{i,k}'\eta_{j,l})=0, \forall i\neq j$ and $k \neq l$, and $\operatorname{E}(\eta_{i,k}'\eta_{j,l})=0, \forall i\neq j$ and $k \neq l$. Therefore, each $x^*_{i,k}$ suffers classical measurement error. Note that $\operatorname{E}(x_{i,k})=\operatorname{E}(x^*_{i,k})=\mu_{x_k}$ and that $\operatorname{V}(x_{i,k})=\sigma^2_{x_k}$ while $\operatorname{V}(x^*_{i,k})=\sigma^2_{x_k}+\sigma^2_{\eta_k}\geq \sigma^2_{x_k}$.

\subsection{Data Generation Process}

Assume that the outcome $y_i$ is determined by the following Data Generation Process (DGP):
\begin{align}
    y_i = \gamma t_i + X_i'\beta + \epsilon_i
\end{align}

where $\g$ is the parameter of the variable of interest $t_i$, $\b=(\b_1,\b_2,\dots \b_p)'$ is the vector of the parameters of the covariates $X_i$ including a constant and $\epsilon_i \sim \operatorname{iid}(0,\sigma^2_\epsilon)$. Under this specification, the coefficients are such that:
\begin{align}
    \left(\begin{array}{l}
{\gamma} \\
{\beta}
\end{array}\right)=\left(\begin{array}{cc}
{\sigma}^2_{t} & \Sigma_{tX} \\
\Sigma_{Xt} & {\Sigma}_{X}
\end{array}\right)^{-1}\left(\begin{array}{c}
\Sigma_{yt} \\
\Sigma_{yX}
\end{array}\right)
\end{align}

Suppose that the econometrician has access to $t_i$ but, instead of $X_i$ she observes $X^*_i$. Then, she specifies the following linear model
\begin{align}
    y_i = \gamma^* t_i + {X^{*}_i}' \beta^* + \zeta_i
\end{align}

the coefficients would be such that
\begin{align}
    \left(\begin{array}{l}
{\gamma}^* \\
{\beta}^*
\end{array}\right)&=\left(\begin{array}{cc}
{\sigma}^2_{t} & \Sigma_{tX^*} \\
\Sigma_{X^*t} & {\Sigma}_{X^*}
\end{array}\right)^{-1}\left(\begin{array}{c}
\Sigma_{yt} \\
\Sigma_{yX^*}
\end{array}\right) \\
& =\left(\begin{array}{cc}
{\sigma}^2_{t} & \Sigma_{tX} \\
\Sigma_{Xt} & {\Sigma}_{X}+{\Sigma}_{\eta}
\end{array}\right)^{-1}\left(\begin{array}{cc}
{\sigma}^2_{t} & \Sigma_{tX} \\
\Sigma_{Xt} & {\Sigma}_{X}
\end{array}\right)\left(\begin{array}{l}
{\gamma} \\
{\beta}
\end{array}\right)
\end{align}

To the see the implications of the of this measurement error in the covariates, consider a simple case where the DGP depends only of the variable of interest and a covariate such that:

\begin{align}
    \left(\begin{array}{l}
{\gamma} \\
{\beta}
\end{array}\right)=\left(\begin{array}{l}
1 \\
1
\end{array}\right)
\end{align}

and with $\sigma^2_t=\Sigma_X=\Sigma_\eta=1$ while $\Sigma_{Xt}=0.6$. Then
\begin{align*}
    \left(\begin{array}{l}
{\gamma}^* \\
{\beta}^*
\end{array}\right)& =\left(\begin{array}{cc}
1 & 0.6 \\
0.6 & 2
\end{array}\right)^{-1}\left(\begin{array}{cc}
1 & 0.6 \\
0.6 & 1
\end{array}\right)\left(\begin{array}{l}
1\\
1
\end{array}\right) \\
\left(\begin{array}{l}
{\gamma}^* \\
{\beta}^*
\end{array}\right)&=\left(\begin{array}{l}
1.37 \\
0.39
\end{array}\right)
\end{align*}

Clearly, both coefficients shows bias when the econometrician assumes a DGP with $X_i^*$: while there is attenuation bias on the coefficient of the covariate, the coefficient of the variable of interest is biased upward given that both variable have positive correlation.

\subsection{Principal Component Regression as bias-correction method}

The classical solution for the measurement-error induced bias in econometrics has been the usage of instrumental variables. Suppose an instrument $Z_i$ that satisfies the relevance condition $\operatorname{E}(Z_i'X_i)\neq 0$ and $\operatorname{E}(Z_i't_i)\neq 0$, and also the exclusion restriction $\operatorname{E}(Z_i'\epsilon_i)=\operatorname{E}(Z_i'\zeta_i)=\operatorname{E}(Z_i'\eta_{i,k})=0$, for all $i$ and $k$. Then premultiplying by $Z_i$ we have
\begin{align}
    Z_i'y_i =  Z_i'\gamma^* t_i +  Z_i'{X^{*}_i}' \beta^* +  Z_i'\zeta_i
\end{align}

and so
\begin{align}
    \left(\begin{array}{l}
{\gamma}^{IV} \\
{\beta}^{IV}
\end{array}\right)
& =\left(\begin{array}{cc}
{\Sigma}_{Zt} & \Sigma_{ZX,Zt} \\
\Sigma_{Zt,ZX}& {\Sigma}_{ZX}+{\Sigma}_{Z\eta}
\end{array}\right)^{-1}\left(\begin{array}{cc}
{\Sigma}_{Zt} & \Sigma_{ZX,Zt} \\
\Sigma_{Zt,ZX} & {\Sigma}_{ZX}
\end{array}\right)\left(\begin{array}{l}
{\gamma} \\
{\beta}
\end{array}\right)\\
& =\left(\begin{array}{cc}
{\Sigma}_{Zt} & \Sigma_{ZX,Zt} \\
\Sigma_{Zt,ZX}& {\Sigma}_{ZX}
\end{array}\right)^{-1}\left(\begin{array}{cc}
{\Sigma}_{Zt} & \Sigma_{ZX,Zt} \\
\Sigma_{Zt,ZX} & {\Sigma}_{ZX}
\end{array}\right)\left(\begin{array}{l}
{\gamma} \\
{\beta}
\end{array}\right) \\
\left(\begin{array}{l}
{\gamma}^{IV} \\
{\beta}^{IV}
\end{array}\right)
& =\left(\begin{array}{l}
{\gamma} \\
{\beta}
\end{array}\right)
\end{align}

However, finding reliable source of exogeneity is difficult, as is proving the exclusion condition. Therefore, the use of IV as a bias-correction method should be taken with care given that its feasibilty is hard.\\

Alternatively, we propose an alternative bias-correction method when there are several missmeasured variables for each covariate, that is when we have more than one $x_{i,k}^*$ for every $x_{i,k}$. Given that in all the miss-measured variables the underlying value is the real value, one could think of extracting the underlying true $x_{i,k}$ through a linear combination of the different $x_{i,k}^*$. Then, we could treat all the $x_{i,k}^*$ as variables that share compononents as follows
\begin{align}
   h_{j}=\underset{h^{\prime} h=1, h^{\prime} h_{1}=0, \ldots, h^{\prime} h_{j-1}=0}{\operatorname{argmax}} \operatorname{var}\left[h^{\prime} X^*_k\right]  
\end{align}


where $h_j$ is the eigenvector of $\Sigma$ associated with the $j^{t h}$ ordered eigenvalue $\lambda_{j}$ of $\Sigma_{X^*_k}$, and the principal components of $X^*_k$ are $U_{j}=h_{j}^{\prime} X^*_k$, where $h_{j}$ is the eigenvector of $\Sigma$ associated with the $j^{t h}$ ordered eigenvalue $\lambda_{j}$ of $\Sigma$.\\

Under our assumptions, the vector of missmeasured values $X^*_k$ of $x_{i,k}$, share only one principal component which is precisely $x_{i,k}$. Then, we only have one principal component, $x_{i,k}$, and so the $x_{i,k}$ is such that
\begin{align}
    x_{i,k}=h_{k}^{\prime} X^*_k
\end{align}

Finally, we could then retrieve the vector of true variables $X_i$
\begin{align}
    X_i=HX^*_i
\end{align}

where $H$ is a matrix such that
\begin{align*}
    H=\left(\begin{array}{ccccc}
h_1 & 0 & 0 & \dots & 0 \\
0 & h_2 & 0 & \dots & 0 \\
\vdots & \ddots & h_3 & \ddots & \vdots \\
0 & \dots & \dots & \dots \ddots & h_p
\end{array}\right)
\end{align*}

and $h_k$ is the vector of eigenvalues for the variable $x_{i,k}$.

Our new linear model then would be
\begin{align}
    y_i = \gamma^{PCR} t_i + H{X^*_i}'\beta^{PCR} + \epsilon_i
\end{align}

where the coefficients are as follows
\begin{align}
    \left(\begin{array}{l}
{\gamma}^{PCR} \\
{\beta}^{PCR}
\end{array}\right)&=\left(\begin{array}{cc}
{\sigma}^2_{t} & \Sigma_{t,HX^*} \\
\Sigma_{HX^*,t} & {\Sigma}_{HX^*}
\end{array}\right)^{-1}\left(\begin{array}{c}
\Sigma_{yt} \\
\Sigma_{y,HX^*}
\end{array}\right)\\
&=\left(\begin{array}{cc}
{\sigma}^2_{t} & \Sigma_{t,HX^*} \\
\Sigma_{HX^*,t} & {\Sigma}_{HX^*}
\end{array}\right)^{-1}\left(\begin{array}{cc}
{\sigma}^2_{t} & \Sigma_{tX} \\
\Sigma_{Xt} & {\Sigma}_{X}
\end{array}\right)\left(\begin{array}{l}
{\gamma} \\
{\beta}
\end{array}\right)\\
&=\left(\begin{array}{l}
{\gamma} \\
{\beta}
\end{array}\right)
\end{align}

where the last equality comes from $(13)$.
\end{document}