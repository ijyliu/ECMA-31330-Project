\documentclass[12pt]{article}
\usepackage{graphicx}
%\usepackage{subcaption}
\usepackage{caption}
\usepackage{mathtools}
\usepackage{tikz,pgfplots}
\usepackage{subfig}
\usepackage{epsfig}
\usepackage{amsmath}
\usepackage{amssymb}
\usepackage[shortlabels]{enumitem}
\usetikzlibrary{angles,patterns,calc}
\usepackage{bbm}
\usepackage{float}
\newcommand\der[2]{\frac{\partial{#1}}{\partial{#2}}}
\DeclareMathOperator*{\argmax}{arg\,max}
\DeclareMathOperator*{\argmin}{arg\,min}

% stuff to put matlab code in 
\usepackage{listings}
\usepackage{color} %red, green, blue, yellow, cyan, magenta, black, white
\definecolor{mygreen}{RGB}{28,172,0} % color values Red, Green, Blue
\definecolor{mylilas}{RGB}{170,55,241}

% Shortcut greek
\def\a{\alpha}
\def\b{\beta}
\def\g{\gamma}
\def\D{\Delta}
\def\d{\delta}
\def\z{\zeta}
\def\k{\kappa}
\def\l{\lambda}
\def\n{\nu}
\def\r{\rho}
\def\s{\sigma}
\def\t{\tau}
\def\x{\xi}
\def\w{\omega}
\def\W{\Omega}

\usepackage[utf8]{inputenc}
\usepackage[english]{babel}
\usepackage{fancyhdr}
\fancypagestyle{firststyle}
{
\fancyhf{}
    \renewcommand{\headrulewidth}{0pt}
   \fancyfoot[C]{\footnotesize Page \thepage\ of \pageref{LastPage}}
}

\newcommand{\numpy}{{\tt numpy}}    % tt font for numpy

\topmargin -.5in
\textheight 9in
\oddsidemargin -.25in
\evensidemargin -.25in
\textwidth 7in

\newcommand{\question}[1]{ \begin{center} \noindent\colorbox{gray!10}{
\parbox{0.8\textwidth}{\vspace{0.125in} #1 \vspace{0.125in} } } \end{center} }

\begin{document}

    \thispagestyle{firststyle}

    \author{Isaac Liu, Nicol\'as Martorell \& Paul Opheim}
    \title{Attenuation Bias, Measurement Error \& Principal Component Analysis} 
    \maketitle

    % code

    \lstset{language=Matlab,%
        %basicstyle=\color{red},
        breaklines=true,%
        morekeywords={matlab2tikz},
        keywordstyle=\color{blue},%
        morekeywords=[2]{1}, keywordstyle=[2]{\color{black}},
        identifierstyle=\color{black},%
        stringstyle=\color{mylilas},
        commentstyle=\color{mygreen},%
        showstringspaces=false,%without this there will be a symbol in the places where there is a space
        numbers=left,%
        numberstyle={\tiny \color{black}},% size of the numbers
        numbersep=9pt, % this defines how far the numbers are from the text
        emph=[1]{for,end,break},emphstyle=[1]\color{red}, %some words to emphasise
        %emph=[2]{word1,word2}, emphstyle=[2]{style},    
    }

    \begin{abstract}

        Shorter version of the abstract (I would say 4-5 sentences in a single paragraph max) goes here
        
    \end{abstract}

    \newpage \clearpage

    % Introduction (no need for a section header for this)

        Many variables of interest in economics are not directly available as empirical data. Instead, economists often use other variables that are imperfect measurements of the true focus of their analysis. These available variables are known as \textit{proxies} or ``variables measured with error'', and, if they suffer from classical measurement error, their use causes \textit{attenuation bias} when they are used as independent variables in econometric estimation. Traditionally, instrumental variables are used as a shock of exogeneity to get rid of this bias, but finding truly exogenous variables that satisfy the exclusion restriction is difficult, and so this method can often not be feasibly applied.

        As an alternative to dealing with attenuation bias, we propose the use of Principal Component Analysis (PCA) over several variables measured with error. When there are multiple observed variables driven by a single ``true'' one, we propose to use PCA over these variables to extract the ``true'' variable. We then use this extracted value and use it in a standard OLS regression, thus providing a solution to attenuation bias that does not require the strong assumptions of instrumental variable analysis.

        To show the properties and behaviour of our estimator on large samples under standard assumptions, we present a theoretical framework and a Monte-Carlo analysis. Additionally, we explore a basic empirical application to our method, by estimating the effect of economic development on life expectancy at birth. Since there is no consensus on how to measure economic development, we take a sample of different variables that may measure economic development with error (GDP per capita, GNI per capita, Household Income Per Capita, among others) over which we apply PCA to apply our identification strategy.

    \section*{Literature}

        Brief discussion of https://warwick.ac.uk/fac/soc/economics/staff/knagasawa/PartialEffects.pdf, as well as anything else important that comes up on Google Scholar

        So it actually kind of seems like we are doing something unique, but it's not clear that what we are doing is better than existing approaches to measurement error. Like averaging or instrumenting variables with each other.

        Nagasawa 2020 
        theoretically develops the use of a proxy variable to deal with unobserved heterogeneity in line with the definitions in the measurement error literature
        Uses an imperfect measurment of the error, the proxy
        problem is in nuisance parameters
        has a single proxy variable
        nonparametric approach mentioned as having limited usefulness due to curse of dimensionality and restrictive common support
        kernel first stage
        new partial effects method of proxy

        Schennach 2016
        focuses on nonclassical measurement error

        https://advstats.psychstat.org/book/factor/index.php#factor-analysis
        factor analysis can handle measurement errors in multiple ways
        exploratory factor analysis can be used to explore dimensionality of measurement instrument (like prof mentioned i guess). I think this is like just the PCA step and looking at the Loadings
        confirmatory factor analysis-test if constructs influence responses
        some unique observation factors
        there is discussion of the variance of the true variable and the variance of the measurmenet error.
        i think there might be misuse of the term measurement errors/this is not clear
        overall not very related

        http://web.pdx.edu/~newsomj/semclass/ho_latent.pdf
        latent variable models... are really basically mathematically the same as measurement error
        simple and standard me explanation
        reminder- standardized coefificetn is very similar to correlation coefficient
        reminder that there is no impact of measurmeent error in y on unstandardized coeff
        multiple regression- x1 coeff can indeed be biased, middle of p.2
        lower statistical power
        no bias in the mean of x
        bad impact on the regression
        use Structural Equation Modelling to estimate path coefficients among latent variables
        need three or more measures to estimate a latent variable
        in a footnote it mentions a case with only two indicators
        not always me but sometimes uniqueness... really it's just items unaccounted for by the factor
        latent variance is independent.. classical assumption

        Wegge
        https://www.sciencedirect.com/science/article/pii/0304407694016763
        measurment error regression models are factor analysis mdoels, with the correct regressors being the factors. indeed. but no common stat method, because true factors not known- no clear coeffiicent linkage. instead, grouping
        dependent variables and latent factors uncorrelated with errors
        counting rules
        data grouping remedies- structural equations
        grouped regression model- ivs and weighted averages of ivs
        if there is very little credible info about the variance of measurment errors or the covariance of equation erros, factor loading restrictions are needed for identification
        this paper is about identification and IVs really

        https://www.ncbi.nlm.nih.gov/pmc/articles/PMC4787301/
        schoefield
        mixed effects structural equations model
        combine structural equations and item response theory
        attenuation, or nonclassical bias in any direction
        clear misestimation consequences
        solutions in IV or nonparametric bounds. here IRT
        error structure assumed by IV often violated
        bayesian structural framework and model
        this paper is focused on ME in the main regressor

        *perhaps note somewhere that governmnet spending is less likely to have measurement error because it's an official statistic... though health spending in the denominator could have error... ow.

    \section*{Theoretical framework}

        Consider a model where the outcome is denoted by $y_i$. This outcome depends on a variable of interest denoted by $t_i$ and a vector of covariates denoted by $X_i=(x_{i,1},x_{i,2},\dots x_{i,p})'$. Additionally, consider a vector of variables $X^*_i=(x^*_{i,1},x^*_{i,2},\dots x^*_{i,p})'$ that correspond to the covariates $X_i$ but observed with measurement error, where $x^*_{i,k}=x_{i,k}+\eta_{i,k}$ with $\eta_{i,k} \sim {iid}(0,\sigma^2_{\eta_k})$, $\operatorname{E}(x_{i,k}'\eta_{i,k})=0, \forall i$, $\operatorname{E}(x_{i,k}'\eta_{j,l})=0, \forall i\neq j$ and $k \neq l$, and $\operatorname{E}(\eta_{i,k}'\eta_{j,l})=0, \forall i\neq j$ and $k \neq l$. Therefore, each $x^*_{i,k}$ suffers from classical measurement error. Note that $\operatorname{E}(x_{i,k})=\operatorname{E}(x^*_{i,k})=\mu_{x_k}$ and that $\operatorname{V}(x_{i,k})=\sigma^2_{x_k}$ while $\operatorname{V}(x^*_{i,k})=\sigma^2_{x_k}+\sigma^2_{\eta_k}\geq \sigma^2_{x_k}$.

    \subsection*{Data Generating Process}

        Assume that the outcome $y_i$ is determined by the following Data Generation Process (DGP):
        \begin{align}
            y_i = \gamma t_i + X_i'\beta + \epsilon_i
        \end{align}

        where $\g$ is the parameter of the variable of interest $t_i$, $\b=(\b_1,\b_2,\dots \b_p)'$ is the vector of the parameters of the covariates $X_i$ including a constant and $\epsilon_i \sim \operatorname{iid}(0,\sigma^2_\epsilon)$. Under this specification, the coefficients are such that:
        \begin{align}
            \left(\begin{array}{l}
        {\gamma} \\
        {\beta}
        \end{array}\right)=\left(\begin{array}{cc}
        {\sigma}^2_{t} & \Sigma_{tX} \\
        \Sigma_{Xt} & {\Sigma}_{X}
        \end{array}\right)^{-1}\left(\begin{array}{c}
        \Sigma_{yt} \\
        \Sigma_{yX}
        \end{array}\right)
        \end{align}

        Suppose that the econometrician has access to $t_i$ but, instead of $X_i$ she observes $X^*_i$. Then, she specifies the following linear model
        \begin{align}
            y_i = \gamma^* t_i + {X^{*}_i}' \beta^* + \zeta_i
        \end{align}

        the coefficients would be such that
        \begin{align}
            \left(\begin{array}{l}
        {\gamma}^* \\
        {\beta}^*
        \end{array}\right)&=\left(\begin{array}{cc}
        {\sigma}^2_{t} & \Sigma_{tX^*} \\
        \Sigma_{X^*t} & {\Sigma}_{X^*}
        \end{array}\right)^{-1}\left(\begin{array}{c}
        \Sigma_{yt} \\
        \Sigma_{yX^*}
        \end{array}\right) \\
        & =\left(\begin{array}{cc}
        {\sigma}^2_{t} & \Sigma_{tX} \\
        \Sigma_{Xt} & {\Sigma}_{X}+{\Sigma}_{\eta}
        \end{array}\right)^{-1}\left(\begin{array}{cc}
        {\sigma}^2_{t} & \Sigma_{tX} \\
        \Sigma_{Xt} & {\Sigma}_{X}
        \end{array}\right)\left(\begin{array}{l}
        {\gamma} \\
        {\beta}
        \end{array}\right)
        \end{align}

        To the see the implications of the of this measurement error in the covariates, consider a simple case where the DGP depends only of the variable of interest and a covariate such that:

        \begin{align}
            \left(\begin{array}{l}
        {\gamma} \\
        {\beta}
        \end{array}\right)=\left(\begin{array}{l}
        1 \\
        1
        \end{array}\right)
        \end{align}

        and with $\sigma^2_t=\Sigma_X=\Sigma_\eta=1$ while $\Sigma_{Xt}=0.6$. Then
        \begin{align*}
            \left(\begin{array}{l}
        {\gamma}^* \\
        {\beta}^*
        \end{array}\right)& =\left(\begin{array}{cc}
        1 & 0.6 \\
        0.6 & 2
        \end{array}\right)^{-1}\left(\begin{array}{cc}
        1 & 0.6 \\
        0.6 & 1
        \end{array}\right)\left(\begin{array}{l}
        1\\
        1
        \end{array}\right) \\
        \left(\begin{array}{l}
        {\gamma}^* \\
        {\beta}^*
        \end{array}\right)&=\left(\begin{array}{l}
        1.37 \\
        0.39
        \end{array}\right)
        \end{align*}

        Clearly, both coefficients shows bias when the econometrician assumes a DGP with $X_i^*$: while there is attenuation bias on the coefficient of the covariate, the coefficient of the variable of interest is biased upward given that some of the effect of the covariates is ``omitted'' given this attenuation.

    \subsection*{Instrumental Variables Regression as a Bias-Correction Method}

        The classical solution for the measurement-error induced bias in econometrics has been the usage of instrumental variables. Suppose an instrument $Z_i$ that satisfies the relevance condition $\operatorname{E}(Z_i'X_i)\neq 0$ and $\operatorname{E}(Z_i't_i)\neq 0$, and also the exclusion restriction $\operatorname{E}(Z_i'\epsilon_i)=\operatorname{E}(Z_i'\zeta_i)=\operatorname{E}(Z_i'\eta_{i,k})=0$, for all $i$ and $k$. Then premultiplying by $Z_i$ we have
        \begin{align}
            Z_i'y_i =  Z_i'\gamma^* t_i +  Z_i'{X^{*}_i}' \beta^* +  Z_i'\zeta_i
        \end{align}

        and so
        \begin{align}
            \left(\begin{array}{l}
        {\gamma}^{IV} \\
        {\beta}^{IV}
        \end{array}\right)
        & =\left(\begin{array}{cc}
        {\Sigma}_{Zt} & \Sigma_{ZX,Zt} \\
        \Sigma_{Zt,ZX}& {\Sigma}_{ZX}+{\Sigma}_{Z\eta}
        \end{array}\right)^{-1}\left(\begin{array}{cc}
        {\Sigma}_{Zt} & \Sigma_{ZX,Zt} \\
        \Sigma_{Zt,ZX} & {\Sigma}_{ZX}
        \end{array}\right)\left(\begin{array}{l}
        {\gamma} \\
        {\beta}
        \end{array}\right)\\
        & =\left(\begin{array}{cc}
        {\Sigma}_{Zt} & \Sigma_{ZX,Zt} \\
        \Sigma_{Zt,ZX}& {\Sigma}_{ZX}
        \end{array}\right)^{-1}\left(\begin{array}{cc}
        {\Sigma}_{Zt} & \Sigma_{ZX,Zt} \\
        \Sigma_{Zt,ZX} & {\Sigma}_{ZX}
        \end{array}\right)\left(\begin{array}{l}
        {\gamma} \\
        {\beta}
        \end{array}\right) \\
        \left(\begin{array}{l}
        {\gamma}^{IV} \\
        {\beta}^{IV}
        \end{array}\right)
        & =\left(\begin{array}{l}
        {\gamma} \\
        {\beta}
        \end{array}\right)
        \end{align}

        However, finding a reliable source of exogeneity is difficult, and it is impossible to conclusively prove a suitable exclusion restriction. The use of IV as a bias-correction method is thus often unfeasible.\\

    \subsection*{Principal Component Regression as Bias-Correction Method}

        Alternatively, we propose an alternative bias-correction method for when there are several mismeasured variables for each covariate; that is, when we have more than one $x_{i,k}^*$ for every $x_{i,k}$. Given that in all the mismeasured variables the underlying value is the real value, one could think of extracting the underlying true $x_{i,k}$ through a linear combination of the different $x_{i,k}^*$. Then, we could treat all the $x_{i,k}^*$ as variables that share components as follows:
        \begin{align}
        h_{j}=\underset{h^{\prime} h=1, h^{\prime} h_{1}=0, \ldots, h^{\prime} h_{j-1}=0}{\operatorname{argmax}} \operatorname{var}\left[h^{\prime} X^*_k\right]  
        \end{align}


        where $h_j$ is the eigenvector of $\Sigma$ associated with the $j^{t h}$ ordered eigenvalue $\lambda_{j}$ of $\Sigma_{X^*_k}$, and the principal components of $X^*_k$ are $U_{j}=h_{j}^{\prime} X^*_k$, where $h_{j}$ is the eigenvector of $\Sigma$ associated with the $j^{t h}$ ordered eigenvalue $\lambda_{j}$ of $\Sigma$.\\

        Under our assumptions, the vector of mismeasured values $X^*_k$ of $x_{i,k}$, share only one principal component which is precisely $x_{i,k}$. Then, we only have one principal component, $x_{i,k}$, and so the $x_{i,k}$ is such that
        \begin{align}
            x_{i,k}=h_{k}^{\prime} X^*_k
        \end{align}

        Finally, we could then retrieve the vector of true variables $X_i$
        \begin{align}
            X_i=HX^*_i
        \end{align}

        where $H$ is a matrix such that
        \begin{align*}
            H=\left(\begin{array}{ccccc}
            h_1 & 0 & 0 & \dots & 0 \\
            0 & h_2 & 0 & \dots & 0 \\
            \vdots & \ddots & h_3 & \ddots & \vdots \\
            0 & \dots & \dots & \dots \ddots & h_p
            \end{array}\right)
        \end{align*}

        and $h_k$ is the vector of eigenvalues for the variable $x_{i,k}$.

        Our new linear model then becomes
        \begin{align}
            y_i = \gamma^{PCR} t_i + H{X^*_i}'\beta^{PCR} + \epsilon_i
        \end{align}

        where the coefficients are as follows
        \begin{align}
            \left(\begin{array}{l}
            {\gamma}^{PCR} \\
            {\beta}^{PCR}
            \end{array}\right)&=\left(\begin{array}{cc}
            {\sigma}^2_{t} & \Sigma_{t,HX^*} \\
            \Sigma_{HX^*,t} & {\Sigma}_{HX^*}
            \end{array}\right)^{-1}\left(\begin{array}{c}
            \Sigma_{yt} \\
            \Sigma_{y,HX^*}
            \end{array}\right)\\
            &=\left(\begin{array}{cc}
            {\sigma}^2_{t} & \Sigma_{t,HX^*} \\
            \Sigma_{HX^*,t} & {\Sigma}_{HX^*}
            \end{array}\right)^{-1}\left(\begin{array}{cc}
            {\sigma}^2_{t} & \Sigma_{tX} \\
            \Sigma_{Xt} & {\Sigma}_{X}
            \end{array}\right)\left(\begin{array}{l}
            {\gamma} \\
            {\beta}
            \end{array}\right)\\
            &=\left(\begin{array}{l}
            {\gamma} \\
            {\beta}
            \end{array}\right)
        \end{align}

        where the last equality comes from $(13)$.

    \section*{Properties of the Estimator: Monte Carlo Simulations}

We then complement our theoretical analysis by using Monte Carlo Simulation to analyze the effects of using Principal Components Regression as a method of bias correction. For these simulations, we assume that the true DGP for the data is:

$$y_i = \beta_1 x_i + \beta_2 z_i + u_i$$

... where $x_i$ and $z_i$ are single variables drawn from $\mathcal{N}(\begin{bmatrix} 0\\ 0 \end{bmatrix}, \begin{bmatrix} 1 & \rho\\ \rho & 1\end{bmatrix})$, where $\rho$ is some covariance between our main variable of interest ($x_i$) and the covariate ($z_i$). The $u_i$ is drawn from a white noise distribution($\mathcal{N}(0,1)$) that is uncorrelated with both $x_i$ and $z_i$. We then assume (as with the theoretical analysis) that $z_i$ is not directly observable and instead the researchers only have access to $p$ many measurements $z_{i,j}^*$ where $z_{i,j}^* = z_i + \eta_j$ where $\eta_j$ is drawn from a white noise distribution $\mathcal{N}(\mathbf{0},\Sigma)$ where $\mathbf{0}$ is a p-vector and $\Sigma$ is a diagonal p by p matrix with only 1s on the diagonal.\\
\\
In our simulations, we assume default values of $\rho = 0.5$, $\beta_1 = \beta_2 = 1$, and $p=5$. We then vary each factor while holding the others fixed, and perform 1,000 simulations of the DGP followed by an OLS regression on either the PCA value from the p measurements of the true $z_i$, or on a single one of the measurements of $z_i$. For each simulation, we generate 100 observations of $y_i,x_i$,etc. Below are the results for different values of $p$:

\begin{table}[!htbp] \centering
  \caption{Average Coefficients for Values of p \label{sim_p_2}}
\begin{tabular}{@{\extracolsep{5pt}}lccccc}
\\[-1.8ex]\hline
\hline \\[-1.8ex]
& \multicolumn{5}{c}{\textit{Number of p}} \
\cr 
\\[-1.8ex] & 5 & 10 & 20 & 50 \\
\hline \\[-1.8ex]
& \multicolumn{5}{c}{\textit{Coefficient on Main Variable}} \\
 PCA & 1.105 & 1.066 & 1.033 & 1.022  \\
  & (0.121) & (0.122) & (0.119) & (0.117)\\
  Single Measurement & 1.280 & 1.283 & 1.282 & 1.292  \\
  & (0.124) & (0.129) & (0.131) & (0.167)\\
& \multicolumn{5}{c}{\textit{Absolute Percentage Error}} \\
  PCA & 13.1\% & 11.1\% & 10.0\% & 9.4\%  \\
   & (9.3 ppts) & (8.3 ppts) & (7.3 ppts) & (7.3 ppts)\\
  Single Measurement & 28.2\% & 28.5\% & 28.3\% & 29.3\%  \\
  & (12.6 ppts) & (12.6 ppts) & (12.6 ppts) & (12.7 ppts)\\
\hline \\[-1.8ex]
 Observations & 1,000 & 1,000 & 1,000 & 1,000 & \\
\hline
\hline \\[-1.8ex]
\end{tabular}
\end{table}

We can see that using PCA to extract the latent covariate driving the mismeasured covariates noticeably outperforms using a single mismeasured covariate across several values of $p$. Both the average coefficient on $\beta_1$ obtained when including the PCA output in the regression, and the mean absolute percentage error obtained on the 1,000 simulations are both much closer to the target values with the PCA-based regression than with the single measurement regression. Additionally, we can see that as $p$ increases the estimated $\beta_1^*$ coefficient in the PCA regression gets steadily closer to the true $\beta_1$ value of 1. Appendix 1 contains charts that show that this increase in performance is also true for different values of $\beta_1$ and $\beta_2$.\\
\\
However, there are certain circumstances where the PCA method does not lead to more accurate estimates of $\beta_1^*$. Let's now look at the simulation results for different values of $\rho$ (the covariance between the main variable of interest $x_i$ and the true latent covariate $z_i$):

\begin{table}[!htbp] \centering
\begin{tabular}{@{\extracolsep{5pt}}lccccc}
\\[-1.8ex]\hline
\hline \\[-1.8ex]
& \multicolumn{5}{c}{\rho \textit{ Value}} \
\cr \cline{5-6}
\\[-1.8ex] & -1 & -0.5 & 0 & 0.5 & 1 \\
\hline \\[-1.8ex]
& \multicolumn{5}{c}{\textit{Coefficient on Main Variable}} \\
 PCA & -0.006 & 0.900 & 0.996 & 1.105 & 2.009  \\
  & (0.238) & (0.120) & (0.111) & (0.121) & (0.242)\\
  Single Measurement & -0.002 & 0.720 & 0.998 & 1.280 & 2.003  \\
  & (0.142) & (0.130) & (0.127) & (0.129) & (0.147)\\
& \multicolumn{5}{c}{\textit{Absolute Percentage Error}} \\
  PCA & 100.6\% & 12.7\% & 8.9\% & 13.1\% & 100.9\% \\
   & (23.8 ppts) & (9.1 ppts) & (6.6 ppts) & (9.3 ppts) & (24.2 ppts)\\
  Single Measurement & 100.2\% & 28.1\% & 10.2\% & 28.2\% & 100.3\%  \\
  & (14.2 ppts) & (12.7 ppts) & (7.6 ppts) & (12.6 ppts) & (14.7 ppts)\\
\hline \\[-1.8ex]
 Observations & 1,000 & 1,000 & 1,000 & 1,000 &\\
\hline
\hline \\[-1.8ex]
\end{tabular}
\end{table}


 When the covariance between $x_i$ and $z_i$ is equal to $0, -1$, or $1$ then there is no notable improvement from using the PCA-extracted latent variable (and notice that since the variances of $x_i$ and $z_i$ are 1, this means that the covariance is equal to the correlation in these simulations). These simulation results suggest that so long as the correlation between $x_i$ and $z_i$ is not close to $-1$,$0$, or $1$, there are noticeable performance gains from using PCA to extract the true covariate from a collection of observed variables that try to measure that true covariate.\\
\\
However, the performance advantages that we see from using PCA could be driven by the benefit of having multiple measurements of our true covariate of interest, as opposed to any special advantages from PCA specifically. We test this question by comparing the estimated $\beta_1^*$ in our PCA regressions with the estimated $\beta_1^*$ when we include all $p$ measurements as separate covariates in the regression, and the $\beta_1^*$ obtained when the covariate is the mean of all $p$ measurements of the true covariate. The results from these regressions for different values of $p$ is shown below:

\begin{table}[!htbp] \centering
  \caption{Average Coefficients for Values of p \label{sim_p_3}}
\begin{tabular}{@{\extracolsep{5pt}}lccccc}
\\[-1.8ex]\hline
\hline \\[-1.8ex]
& \multicolumn{5}{c}{\textit{Number of p}} \
\cr 
\\[-1.8ex] & 5 & 10 & 20 & 50 \\
\hline \\[-1.8ex]
& \multicolumn{5}{c}{\textit{Coefficient on Main Variable}} \\
 PCA & 1.105 & 1.066 & 1.033 & 1.022  \\
  & (0.121) & (0.122) & (0.119) & (0.117)\\
 All Measurements & 1.100 & 1.061 & 1.025 & 1.010  \\
  & (0.124) & (0.129) & (0.131) & (0.167)\\
 Average of Measurements & 1.100 & 1.060 & 1.026 & 1.015  \\
  & (0.121) & (0.122) & (0.119) & (0.117)\\
& \multicolumn{5}{c}{\textit{Absolute Percentage Error}} \\
  PCA & 13.1\% & 11.1\% & 10.0\% & 9.4\%  \\
   & (9.3 ppts) & (8.3 ppts) & (7.3 ppts) & (7.3 ppts)\\
  All Measurements & 12.9\% & 11.4\% & 10.7\% & 13.2\%  \\
  & (9.3 ppts) & (8.5 ppts) & (7.9 ppts) & (10.2 ppts)\\
  Average of Measurements & 12.8\% & 10.9\% & 9.8\% & 9.3\%  \\
  & (9.2 ppts) & (8.2 ppts) & (7.2 ppts) & (7.2 ppts)\\
\hline \\[-1.8ex]
 Observations & 1,000 & 1,000 & 1,000 & 1,000 & \\
\hline
\hline \\[-1.8ex]
\end{tabular}
\end{table}

As one can see from these results (and results for different values of $\beta_1$, $\beta_2$, and $\rho$ in Appendix 2), there does not seem to be a noticeable difference between these three regression methods (across any values of $p, \beta_1,\beta_2$, and $\rho$. Thus, our simulations suggest that there are major benefits to having multiple measurements of a latent covariate of interest, but that using PCA, taking the average of these measurements, and including all measurements as separate covariates seem to give similar benefits to the performance of the regression.


    \section*{Application: Government Share of Healthcare Spending and Life Expectancy}

        Explain economic importance/interesting-ness of the chosen application

        Explain how GDP/economic development is measured with error

        It is very difficult to find an instrumental variable for economic development which satisfies a reasonable exclusion restriction.

        % Add \ref to the following table?
        In the left column in the table below I first regress the life expectancy at birth for all individuals in a given country and year on a measure of government spending as a share of total health expenditure. In the middle column I include the economic controls/covariates of GDP per capita (PPP), GNI per capita (PPP), Survey Mean Income/Consumption Per Capita, ILO GDP per person employed, and Net Foreign Assets Per Capita, all from the World Bank. In the rightmost column I instead use the first principal component combining these covariates.

        I standardize all variables by subtracting the mean and dividing by the standard deviation, linearly interpolate data between known observations, and remove country-years with missing values for any of the economic indicators.

        \begin{table}[!htbp] \centering
\begin{tabular}{@{\extracolsep{5pt}}lccccc}
\\[-1.8ex]\hline
\hline \\[-1.8ex]
& \multicolumn{5}{c}{\textit{Life Expectancy at Birth (Years)}} \
\cr \cline{5-6}
\\[-1.8ex] & (1) & (2) & (3) & (4) & (5) \\
\hline \\[-1.8ex]
 Govt. Share of Health Exp. & 0.564$^{***}$ & 0.256$^{***}$ & 0.024$^{***}$ & 0.298$^{***}$ & 0.026$^{***}$ \\
  & (0.018) & (0.018) & (0.008) & (0.018) & (0.007) \\
 Covariates & None & Econ Indicators & Econ Indicators & PCs & PCs \\
 Fixed Effects & No & No & Yes & No & Yes \\
\hline \\[-1.8ex]
 Observations & 1,995 & 1,995 & 1,995 & 1,995 & 1,995 \\
 $R^2$ & 0.319 & 0.573 & 0.987 & 0.518 & 0.987 \\
 Adjusted $R^2$ & 0.318 & 0.569 & 0.985 & 0.517 & 0.985 \\
 Residual Std. Error & 0.826 & 0.656 & 0.121 & 0.695 & 0.121  \\
 F Statistic & 931.676$^{***}$  & 176.737$^{***}$  & 2458.091$^{***}$  & 1069.682$^{***}$  & 413673698880527.375$^{***}$  \\
\hline
\hline \\[-1.8ex]
\textit{Note:} & \multicolumn{5}{r}{$^{*}$p$<$0.1; $^{**}$p$<$0.05; $^{***}$p$<$0.01} \\
 & \multicolumn{5}{r}\textit{All variables are standardized.} \\
\end{tabular}
\end{table}

        % Though the coefficients are not readily interpretable, they do differ from each other

        % ADD TEST to show coefficients differ? I think this also would not be easy to interpret since the independent variables are different...

        % In both cases, they are significant.

        % Notably, we demonstrate a higher $R^2$ using the principal component model.

    \section*{Conclusion}

    \clearpage \newpage

    \appendix

    \section*{Appendix 1}

        \begin{table}[!htbp] \centering
\begin{tabular}{@{\extracolsep{5pt}}lccccc}
\\[-1.8ex]\hline
\hline \\[-1.8ex]
& \multicolumn{5}{c}{\textit{True $\beta_1$}} \
\cr 
\\[-1.8ex] & 0.1 & 1 & 10 & 100 \\
\hline \\[-1.8ex]
& \multicolumn{5}{c}{\textit{Coefficient on Main Variable}} \\
 PCA & 0.207 & 1.105 & 10.104 & 100.117  \\
  & (0.121) & (0.121) & (0.123) & (0.124)\\
  Single Measurement & 0.383 & 1.280 & 10.278 & 100.289  \\
  & (0.128) & (0.129) & (0.131) & (0.133)\\
& \multicolumn{5}{c}{\textit{Absolute Percentage Error}} \\
  PCA & 131.1\% & 13.1\% & 1.3\% & 0.1\%  \\
   & (95.1 ppts) & (9.3 ppts) & (0.9 ppts) & (0.1 ppts)\\
  Single Measurement & 283.6\% & 28.2\% & 2.8\% & 0.3\%  \\
  & (126.6 ppts) & (12.6 ppts) & (1.3 ppts) & (0.1 ppts)\\
\hline \\[-1.8ex]
 Observations & 1,000 & 1,000 & 1,000 & 1,000 &\\
\hline
\hline \\[-1.8ex]
\end{tabular}
\end{table}
        \begin{table}[!htbp] \centering
  \caption{Average Coefficients for Values of $\beta_2$ \label{sim_beta2_2}}
\begin{tabular}{@{\extracolsep{5pt}}lccccc}
\\[-1.8ex]\hline
\hline \\[-1.8ex]
& \multicolumn{5}{c}{\textit{True $\beta_2$}} \
\cr 
\\[-1.8ex] & 0.1 & 1 & 10 & 100 \\
\hline \\[-1.8ex]
& \multicolumn{5}{c}{\textit{Coefficient on Main Variable}} \\
 PCA & 1.018 & 1.105 & 2.112 & 12.171  \\
  & (0.115) & (0.121) & (0.477) & (4.555)\\
  Single Measurement & 1.034 & 1.280 & 3.865 & 29.751  \\
  & (0.107) & (0.129) & (0.703) & (7.231)\\
& \multicolumn{5}{c}{\textit{Absolute Percentage Error}} \\
  PCA & 9.4\% & 13.1\% & 111.6\% & 1,119.6\%  \\
   & (7.0 ppts) & (9.3 ppts) & (47.0 ppts) & (449.4 ppts)\\
  Single Measurement & 8.9\% & 28.2\% & 286.5\% & 2,875.1\%  \\
  & (6.8 ppts) & (12.6 ppts) & (70.3 ppts) & (723.1 ppts)\\
\hline \\[-1.8ex]
 Observations & 1,000 & 1,000 & 1,000 & 1,000 &\\
\hline
\hline \\[-1.8ex]
\end{tabular}
\end{table}
        \begin{table}[!htbp] \centering
  \caption{Average Coefficients for Values of p \label{sim_p_2}}
\begin{tabular}{@{\extracolsep{5pt}}lccccc}
\\[-1.8ex]\hline
\hline \\[-1.8ex]
& \multicolumn{5}{c}{\textit{Number of p}} \
\cr 
\\[-1.8ex] & 5 & 10 & 20 & 50 \\
\hline \\[-1.8ex]
& \multicolumn{5}{c}{\textit{Coefficient on Main Variable}} \\
 PCA & 1.105 & 1.066 & 1.033 & 1.022  \\
  & (0.121) & (0.122) & (0.119) & (0.117)\\
  Single Measurement & 1.280 & 1.283 & 1.282 & 1.292  \\
  & (0.124) & (0.129) & (0.131) & (0.167)\\
& \multicolumn{5}{c}{\textit{Absolute Percentage Error}} \\
  PCA & 13.1\% & 11.1\% & 10.0\% & 9.4\%  \\
   & (9.3 ppts) & (8.3 ppts) & (7.3 ppts) & (7.3 ppts)\\
  Single Measurement & 28.2\% & 28.5\% & 28.3\% & 29.3\%  \\
  & (12.6 ppts) & (12.6 ppts) & (12.6 ppts) & (12.7 ppts)\\
\hline \\[-1.8ex]
 Observations & 1,000 & 1,000 & 1,000 & 1,000 & \\
\hline
\hline \\[-1.8ex]
\end{tabular}
\end{table}
        \begin{table}[!htbp] \centering
\begin{tabular}{@{\extracolsep{5pt}}lccccc}
\\[-1.8ex]\hline
\hline \\[-1.8ex]
& \multicolumn{5}{c}{\rho \textit{ Value}} \
\cr \cline{5-6}
\\[-1.8ex] & -1 & -0.5 & 0 & 0.5 & 1 \\
\hline \\[-1.8ex]
& \multicolumn{5}{c}{\textit{Coefficient on Main Variable}} \\
 PCA & -0.006 & 0.900 & 0.996 & 1.105 & 2.009  \\
  & (0.238) & (0.120) & (0.111) & (0.121) & (0.242)\\
  Single Measurement & -0.002 & 0.720 & 0.998 & 1.280 & 2.003  \\
  & (0.142) & (0.130) & (0.127) & (0.129) & (0.147)\\
& \multicolumn{5}{c}{\textit{Absolute Percentage Error}} \\
  PCA & 100.6\% & 12.7\% & 8.9\% & 13.1\% & 100.9\% \\
   & (23.8 ppts) & (9.1 ppts) & (6.6 ppts) & (9.3 ppts) & (24.2 ppts)\\
  Single Measurement & 100.2\% & 28.1\% & 10.2\% & 28.2\% & 100.3\%  \\
  & (14.2 ppts) & (12.7 ppts) & (7.6 ppts) & (12.6 ppts) & (14.7 ppts)\\
\hline \\[-1.8ex]
 Observations & 1,000 & 1,000 & 1,000 & 1,000 &\\
\hline
\hline \\[-1.8ex]
\end{tabular}
\end{table}

        
    \section*{Appendix 2}


        \begin{table}[!htbp] \centering
  \caption{Average Coefficients for Values of $\beta_1$ \label{sim_beta1_3}}
\begin{tabular}{@{\extracolsep{5pt}}lccccc}
\\[-1.8ex]\hline
\hline \\[-1.8ex]
& \multicolumn{5}{c}{\textit{True $\beta_1$}} \
\cr 
\\[-1.8ex] & 0.1 & 1 & 10 & 100 \\
\hline \\[-1.8ex]
& \multicolumn{5}{c}{\textit{Coefficient on Main Variable}} \\
 PCA & 0.207 & 1.105 & 10.104 & 100.117  \\
  & (0.121) & (0.121) & (0.123) & (0.124)\\
 All Measurements & 0.201 & 1.100 & 10.098 & 100.11  \\
  & (0.123) & (0.124) & (0.126) & (0.127)\\
 Average of Measurements & 0.202 & 1.100 & 10.098 & 100.111  \\
  & (0.121) & (0.121) & (0.123) & (0.124)\\
& \multicolumn{5}{c}{\textit{Absolute Percentage Error}} \\
  PCA & 131.1\% & 13.1\% & 1.3\% & 0.1\%  \\
   & (95.1 ppts) & (9.3 ppts) & (0.9 ppts) & (0.1 ppts)\\
All Measurements & 128.9\% & 12.9\% & 1.3\% & 0.1\%  \\
  & (93.6 ppts) & (9.3 ppts) & (1.0 ppts) & (0.1 ppts)\\
  Average of Measurements & 127.9\% & 12.8\% & 1.3\% & 0.1\%  \\
  & (93.1 ppts) & (9.2 ppts) & (0.9 ppts) & (0.1 ppts)\\
\hline \\[-1.8ex]
 Observations & 1,000 & 1,000 & 1,000 & 1,000 &\\
\hline
\hline \\[-1.8ex]
\end{tabular}
\end{table}
        \begin{table}[!htbp] \centering
  \caption{Average Coefficients for Values of $\beta_2$ \label{sim_beta2_3}}
\begin{tabular}{@{\extracolsep{5pt}}lccccc}
\\[-1.8ex]\hline
\hline \\[-1.8ex]
& \multicolumn{5}{c}{\textit{True $\beta_2$}} \
\cr 
\\[-1.8ex] & 0.1 & 1 & 10 & 100 \\
\hline \\[-1.8ex]
& \multicolumn{5}{c}{\textit{Coefficient on Main Variable}} \\
 PCA & 1.018 & 1.105 & 2.112 & 12.171  \\
  & (0.115) & (0.121) & (0.477) & (4.555)\\
 All Measurements & 1.02 & 1.100 & 2.067 & 11.664  \\
  & (0.118) & (0.124) & (0.477) & (4.519)\\
 Average of Measurements & 1.017 & 1.100 & 2.061 & 11.625  \\
  & (0.115) & (0.121) & (0.470) & (4.415)\\
& \multicolumn{5}{c}{\textit{Absolute Percentage Error}} \\
  PCA & 9.4\% & 13.1\% & 111.6\% & 1,119.6\%  \\
   & (7.0 ppts) & (9.3 ppts) & (47.0 ppts) & (449.4 ppts)\\
All Measurements & 9.7\% & 12.9\% & 107.1\% & 1,069.3\%  \\
  & (7.0 ppts) & (9.3 ppts) & (46.8 ppts) & (445.0 ppts)\\
  Average of Measurements & 9.4\% & 12.8\% & 106.5\% & 1,065.2\%  \\
  & (6.9 ppts) & (9.2 ppts) & (46.0 ppts) & (435.0 ppts)\\
\hline \\[-1.8ex]
 Observations & 1,000 & 1,000 & 1,000 & 1,000 &\\
\hline
\hline \\[-1.8ex]
\end{tabular}
\end{table}

        \begin{table}[!htbp] \centering
  \caption{Average Coefficients for Values of p \label{sim_p_3}}
\begin{tabular}{@{\extracolsep{5pt}}lccccc}
\\[-1.8ex]\hline
\hline \\[-1.8ex]
& \multicolumn{5}{c}{\textit{Number of p}} \
\cr 
\\[-1.8ex] & 5 & 10 & 20 & 50 \\
\hline \\[-1.8ex]
& \multicolumn{5}{c}{\textit{Coefficient on Main Variable}} \\
 PCA & 1.105 & 1.066 & 1.033 & 1.022  \\
  & (0.121) & (0.122) & (0.119) & (0.117)\\
 All Measurements & 1.100 & 1.061 & 1.025 & 1.010  \\
  & (0.124) & (0.129) & (0.131) & (0.167)\\
 Average of Measurements & 1.100 & 1.060 & 1.026 & 1.015  \\
  & (0.121) & (0.122) & (0.119) & (0.117)\\
& \multicolumn{5}{c}{\textit{Absolute Percentage Error}} \\
  PCA & 13.1\% & 11.1\% & 10.0\% & 9.4\%  \\
   & (9.3 ppts) & (8.3 ppts) & (7.3 ppts) & (7.3 ppts)\\
  All Measurements & 12.9\% & 11.4\% & 10.7\% & 13.2\%  \\
  & (9.3 ppts) & (8.5 ppts) & (7.9 ppts) & (10.2 ppts)\\
  Average of Measurements & 12.8\% & 10.9\% & 9.8\% & 9.3\%  \\
  & (9.2 ppts) & (8.2 ppts) & (7.2 ppts) & (7.2 ppts)\\
\hline \\[-1.8ex]
 Observations & 1,000 & 1,000 & 1,000 & 1,000 & \\
\hline
\hline \\[-1.8ex]
\end{tabular}
\end{table}
        \begin{table}[!htbp] \centering
\begin{tabular}{@{\extracolsep{5pt}}lccccc}
\\[-1.8ex]\hline
\hline \\[-1.8ex]
& \multicolumn{5}{c}{\rho \textit{ Value}} \
\cr \cline{5-6}
\\[-1.8ex] & -1 & -0.5 & 0 & 0.5 & 1 \\
\hline \\[-1.8ex]
& \multicolumn{5}{c}{\textit{Coefficient on Main Variable}} \\
 PCA & -0.006 & 0.900 & 0.996 & 1.105 & 2.009  \\
  & (0.238) & (0.120) & (0.111) & (0.121) & (0.242)\\
 All Measurements & -0.007 & 0.904 & 0.996 & 1.100 & 2.011  \\
  & (0.249) & (0.122) & (0.112) & (0.124) & (0.249)\\
 Average of Measurements & -0.005 & 0.905 & 0.996 & 1.100 & 2.010  \\
  & (0.243) & (0.120) & (0.110) & (0.121) & (0.246)\\
& \multicolumn{5}{c}{\textit{Absolute Percentage Error}} \\
  PCA & 100.6\% & 12.7\% & 8.9\% & 13.1\% & 100.9\% \\
   & (23.8 ppts) & (9.1 ppts) & (6.6 ppts) & (9.3 ppts) & (24.2 ppts)\\
All Measurements & 100.7\% & 12.6\% & 9.0\% & 12.9\% & 101.1\%  \\
  & (24.9 ppts) & (9.0 ppts) & (6.7 ppts) & (9.3 ppts) & (24.9 ppts)\\
  Average of Measurements & 100.5\% & 12.4\% & 8.9\% & 12.8\% & 101.0\% \\
  & (24.3 ppts) & (8.9 ppts) & (6.6 ppts) & (9.2 ppts) & (24.6 ppts)\\
  \hline \\[-1.8ex]
 Observations & 1,000 & 1,000 & 1,000 & 1,000 &\\
\hline
\hline \\[-1.8ex]
\end{tabular}
\end{table}


    \section*{Appendix 3}

        \begin{figure}[h!]
            \centering
            \caption{Correlations Between Covariates and Life Expectancy}
            \label{LE_Health_Econ_Correlations}	
            \includegraphics[width=\linewidth,keepaspectratio=true]{../Output/Figures/LE_Health_Econ_Correlations.pdf}
        \end{figure}

        \begin{figure}[h!]
            \centering
            \caption{Economic Measures PCA Loadings}
            \label{Econ_Loadings}	
            \includegraphics[width=\linewidth,keepaspectratio=true]{../Output/Figures/Econ_Indicator_Loadings.pdf}
        \end{figure}

        \begin{figure}[h!]
            \centering
            \caption{Economic Measures PCA Share of Variance Explained}
            \label{Econ_Share_Explained}	
            \includegraphics[width=\linewidth,keepaspectratio=true]{../Output/Figures/Econ_Indicator_Share_Explained.pdf}
        \end{figure}

\end{document}